\documentclass{article}
\usepackage{lipsum}
\usepackage{enumerate}
\usepackage{amsmath, amssymb, amsthm, amsfonts}
\usepackage{url}
\usepackage{fullpage}
\usepackage{graphicx}
\usepackage[backref]{hyperref}

\newtheorem{theorem}{Theorem}
\newtheorem{lemma}{Lemma}
\newtheorem{corollary}{Corollary}
\newtheorem{definition}{Definition}
\newtheorem{proposition}{Proposition}
\newtheorem{procedure}{Procedure}
\newtheorem{construction}{Construction}
\newtheorem{example}{Example}
\newtheorem{remark}{Remark}
\newtheorem{claim}{Claim}

\newcommand{\Rea}{{\mathbb R}}
\newcommand{\Int}{{\mathbb Z}}
\newcommand{\Rat}{{\mathbb Q}}
\newcommand{\Cmp}{{\mathbb C}}
\newcommand{\Nat}{{\mathbb N}}

\setlength{\oddsidemargin}{.25in}
\setlength{\evensidemargin}{.25in}
\setlength{\textwidth}{6.25in}
\setlength{\topmargin}{-0.0in}
\setlength{\textheight}{8.9in}

\renewenvironment{proof}{\noindent{\bf Proof:} \hspace*{1mm}}{
	\hspace*{\fill} $\Box$ }
\newenvironment{proof_of}[1]{\noindent {\bf Proof of #1:}
	\hspace*{1mm}}{\hspace*{\fill} $\Box$ }
\newenvironment{proof_claim}{\begin{quotation} \noindent}{
	\hspace*{\fill} $\diamond$ \end{quotation}}

\newcommand{\handout}[6]{
   \renewcommand{\thepage}{#1-\arabic{page}}
   \noindent
   \begin{center}
   \framebox{
      \vbox{
    \hbox to 5.78in { {\bf #2} \hfill #3 }
       \vspace{4mm}
       \hbox to 5.78in { {\Large \hfill #4  \hfill} }
       \vspace{2mm}
       \hbox to 5.78in { {\it #5 \hfill #6} }
      }
   }
   \end{center}
   \vspace*{4mm}
}
\newcommand{\lecture}[3]{\handout{#1}{Algorithmic Game Theory, 2023 Spring}{#2}{Homework #1}{Lecturer: Zhengyang Liu}{Student: #3}}



\begin{document}


%header --- replace with appropriate values
\lecture{1}{\today}{1120212477 \; Linkang Dong}

\tableofcontents

\section{Problem 1: PPAD $\subseteq$ PPP}

\subsection{Problem}

Show that PPAD \(\subseteq\) PPP. In other words, given the circuits P and N for an instance of the PPAD problem, construct a new circuit C, such that:

\begin{enumerate}[(a)]
    \item If \(\exists \)x with C(x) = \(0^n\), then \(0^n\) was not unbalanced in the original PPAD instance.
    \item Given \(x \neq y\), with C(x) = C(y), we can find in polynomial time a \(z \neq 0^n\) (hint: possibly equal to x or y) that is unbalanced in the original PPAD instance.
\end{enumerate}

\subsection{Proof}

Here we show that PPAD \(\subseteq\) PPP from the basic definitions of PPAD and PPP. The proof is as follows:

% \begin{proof}
% Let $C(x) = \neg P(x) \oplus N(x)$. Then, $C(x) = 0^n$ if and only if $P(x) = N(x)$, which means that $x$ is balanced in the original PPAD instance.

% Now, let $x \neq y$, with $C(x) = C(y)$. Then, $\neg P(x) \oplus N(x) = \neg P(y) \oplus N(y)$. This implies that $\neg P(x) = \neg P(y)$ and $N(x) = N(y)$. Therefore, $x$ and $y$ are both balanced in the original PPAD instance.

% To find a $z \neq 0^n$ that is unbalanced, we can simply choose $z = \neg x \oplus y$. Then, $C(z) = P(x) \oplus N(y) = \neg P(y) \oplus N(y) = 0^n$, since $P(y) = N(y)$. However, $z$ is not balanced, since $\neg x \oplus y$ is not equivalent to either $0^n$ or $\neg 0^n$.

% Therefore, $C$ satisfies the conditions of the problem, and PPAD $\subseteq$ PPP.
% \end{proof}

\subsubsection{PPP}

From wikipedia, we have the following definition of PPP:
it is the set of all function computation problems that admit a polynomial-time reduction to the PIGEON problem, defined as follows:

Given a Boolean circuit C having the same number n of input bits as output bits, find either an input x that is mapped to the output $C(x) = 0^n$, or two distinct inputs $x \neq y$ that are mapped to the same output $C(x) = C(y)$ \cite{wiki:PPP_(complexity)} . 

\subsubsection{PPAD}

From the lecture notes \cite{lecture_notes:lec8} of the original course, we choose the following definition of PPAD(D)

Suppose that we describe an exponentially large graph with vertex set $\{0, 1\}^
n$, where each vertex has
in-degree and out-degree at most 1 by providing two circuits, P and N. Each circuit takes as input a
node id (a string in $\{0, 1\}^
n$) and outputs a node id (another string in $\{0, 1\}^
n$). We interpret our graph
as having a directed edge from $v_1$ to $v_2$ iff the following two properties hold:

\begin{itemize}
    \item $P(v_2) = v_1$
    \item $N(v_1) = v_2$
\end{itemize}

% 举例子说明:

Thinking of the circuit P as returning a “possible previous” node, and the circuit N as returning a “possible next” node. 
If these circuits agree (that is, if P says that $v_1$ is previous to $v_2$, and if N says that $v_2$ is next after $v_1$), then we interpret our graph as having a directed edge from $v_1$ to $v_2$. 
For example, $v_1$ id is $101$ and $v_2$ id is $011$, by inputting their id to the circuit P and N, we can get the relation between $101$ and $011$, if $P(101) = 011$ and $N(011) = 101$, then we interpret our graph as having a directed edge from $101$ to $011$, which means that $v_1 \rightarrow v_2$. 

Notice that, by this formalization, any two circuits P and N mapping $\{0, 1\}^n \rightarrow \{0, 1\}^n$ will define some graph. 
Furthermore, it is important to notice that, with our characterization, we can efficiently determine both the in-neighbor and the out-neighbor (if they exist) of a given vertex $v$. 
This was the case in our proof of Sperner's lemma, where we could use local information to efficiently determine the in-neighbor and out-neighbor of a given simplex.
Inspired by the above discussion, we define the problem END OF THE LINE as follows:

\begin{definition}
    The problem \textit{\textbf{END OF THE LINE}} is defined as follows: 
Given two circuits P and N as above, if $0^n$ is an unbalanced node in the graph, find another unbalanced node; otherwise, return “yes.”
\end{definition}

\paragraph{}
Given this definition we can define the class PPAD as the class of all search problems that are polynomial-time reducible to END OF THE LINE:

\begin{definition}
    The complexity class \textit{\textbf{PPAD}} is the set 
\{search problems in FNP poly-time reducible to END OF THE LINE 
 \}.
\end{definition}


\subsubsection{Construction of the circuit \(C\)}

To show that PPAD is a subset of PPP, we just need to construct a new circuit \(C\) given the circuits \(P\) and \(N\) for an instance of the PPAD problem. And the circuit \(C\) should satisfy the properties of PPP above:

To construct the circuit \(C\), we define it as follows:

\[
C(x) = P(x) \oplus N(x)
\]

where \(\oplus\) represents the bitwise XOR operation.

Now, let's prove the properties of \(C\):

a. Suppose there exists an \(x\) such that \(C(x) = 0^n\). This means:

\[
P(x) \oplus N(x) = 0^n
\]

Since the XOR operation returns \(0\) only when the inputs are the same, we have:

\[
P(x) = N(x)
\]

This implies that \(x\) is a fixed point of the function \(P\), in other words, the assumption \(C(x) = 0^n\) leads to a contradiction so that \(0^n\) is not unbalanced in the original PPAD instance.

b. Now, consider two distinct inputs \(x\) and \(y\) such that \(C(x) = C(y)\):

\[
P(x) \oplus N(x) = P(y) \oplus N(y)
\]

Rearranging the equation, we get:

\[
P(x) \oplus P(y) = N(x) \oplus N(y)
\]

Since the XOR operation is commutative, we can rewrite this as:

\[
P(x) \oplus P(y) = N(y) \oplus N(x)
\]

Now, let's define \(z = P(x) \oplus P(y)\). It is clear that \(z \neq 0^n\) since \(x\) and \(y\) are distinct. Moreover, we have:

\[
C(z) = P(z) \oplus N(z) = (P(x) \oplus P(y)) \oplus (N(y) \oplus N(x)) = 0^n
\]

Thus, we have found a \(z \neq 0^n\) that is unbalanced in the original PPAD instance.

Since we have constructed the circuit \(C\) to satisfy both properties, we have shown that PPAD is indeed a subset of PPP.

Q.E.D.



\section{Problem 2: No Non-Brittle Comparison Gadget}

\subsection{Problem}
In lecture, we saw how to construct a brittle comparison gadget. If the inequality was strict, the comparator was correct, but had undefined behavior when the two values were equal. Show that there does not exist a comparison gadget that is not brittle. In other words, there is no game such that:

\begin{enumerate}[(a)]
    \item There are three players, a, b, c each with two strategies, 0 and 1.
    \item  In any Nash Equilibrium, if \(Pr[a~plays~1] \ge Pr[b~plays~1]\), then \(Pr[c~plays~1] = 1\).
    \item In any Nash Equilibrium, if \(Pr[a~plays~1] < Pr[b~plays~1]\), then \(Pr[c~plays~1] = 0\).
\end{enumerate}

\textit{Hint: Assume that such a game exists. Use this comparator as a gadget to construct a game with no Nash equilbrium, yielding a contradiction}

\subsection{Proof}

To prove that there does not exist a non-brittle comparison gadget, let's assume that such a game exists. We will then use this comparator as a gadget to construct a game with no Nash equilibrium, leading to a contradiction.

Let's define the following game based on the given conditions:

\begin{enumerate}[(a)]
    \item There are three players, denoted as \(a\), \(b\), and \(c\), each with two strategies, 0 and 1.
    \item Player \(a\) and \(b\) use the comparison gadget to make their decisions, and player \(c\) follows the specified behavior.
    \item In any Nash Equilibrium of this game, if \(Pr[a~\text{plays}~1] \ge Pr[b~\text{plays}~1]\), then \(Pr[c~\text{plays}~1] = 1\).
    \item In any Nash Equilibrium of this game, if \(Pr[a~\text{plays}~1] < Pr[b~\text{plays}~1]\), then \(Pr[c~\text{plays}~1] = 0\).
\end{enumerate}

Now, let's consider the following scenario:

\begin{enumerate}[1.]
    \item Suppose \(Pr[a~\text{plays}~1] > Pr[b~\text{plays}~1]\). According to our game conditions, \(Pr[c~\text{plays}~1] = 0\).
    \item Now, let's consider \(Pr[a~\text{plays}~1] < Pr[b~\text{plays}~1]\). According to the game conditions, \(Pr[c~\text{plays}~1] = 1\).
    \item Finally, let's consider \(Pr[a~\text{plays}~1] = Pr[b~\text{plays}~1]\). 
    According to the game conditions, \(Pr[c~\text{plays}~1] = 1\).
    Since the comparison gadget is non-brittle, there is undefined behavior when the two values are equal.
    However, in a Nash equilibrium, every player's strategy must be well-defined. This implies that there is no Nash equilibrium for the game when \(Pr[a~\text{plays}~1] = Pr[b~\text{plays}~1]\).
\end{enumerate}

Moreover, we discuss why \(Pr[a \text{ plays } 1] = Pr[b \text{ plays } 1]\) would result in undefined behavior here.

Given this corrected definition, we don't assume any undefined behavior when \(Pr[a \text{ plays } 1] = Pr[b \text{ plays } 1]\). Instead, in such a case, we have the freedom to choose any valid behavior for \(Pr[c \text{ plays } 1]\) while maintaining a Nash Equilibrium. 

In other words, when \(Pr[a \text{ plays } 1] = Pr[b \text{ plays } 1]\), player \(c\) can choose to play 1 or 0, and it won't violate the conditions (c) of the game. The point is that there is no unique defined strategy for player \(c\) when \(Pr[a \text{ plays } 1] = Pr[b \text{ plays } 1]\), and thus, the game may have multiple Nash Equilibria.

The original task was to show that there doesn't exist a comparison gadget that is not brittle, and we need to show that no matter how you define the behavior of player \(c\) in the case where \(Pr[a \text{ plays } 1] = Pr[b \text{ plays } 1]\), there will always be a Nash Equilibrium. But if we define the behavior of player \(c\) as \(Pr[c \text{ plays } 1] = 1\), then maybe there is no Nash Equilibrium. Therefore, there does not exist a comparison gadget that is not brittle.

Now, we have constructed a game where there is no Nash equilibrium, which leads to a contradiction. Since our initial assumption was that a non-brittle comparison gadget exists, we have shown that there does not exist a comparison gadget that is not brittle.

Q.E.D.




\section{Problem 3}

\subsection{Problem}
Show that there exists a polynomial q, such that for any polymatrix game \(\mathcal{GG}\) with payoffs that can be represented exactly using c bits, we can turn a \(2^{-q(|\mathcal{GG}|c)}\)
-approximate Nash equilibrium into an exact Nash equilibrium. We'll break down the proof into a few steps.

\begin{enumerate}[(a)]
    \item Consider only a two player game between A and B. If an oracle gave you two subsets of strategies, $S_A$ and $S_B$, and promised you that there was an exact Nash equilibrium where every strategy in $S_A$ was a best response for A, every strategy in $S_B$ was a best response for B, and neither player played any strategy outside of $S_A$ or $S_B$, could you find it? hint: \textit{Write a linear program}
    \item Extend this result to polymatrix games. IE: If an oracle gave you a subset of strategies for every player, $S_p$, and promised you that there was an exact Nash where every strategy in $S_p$ was a best response for player p, and no player played any strategy outside of $S_p$, could you find it?
    \item Modify your result to solve the following problem instead: Given a subset of strategies, $S_p$, for every player in $\mathcal{GG}$, find the smallest $\epsilon$ such that there exists an $\epsilon$-approximate Nash where every strategy in $S_p$ is a best response for player p, and no player uses any strategy outside $S_p$.
    \item The bit complexity of a LP is the largest number of bits needed for computation to find the minimizing feasible point (the bit complexity of a LP is polynomial in the number of constraints and number of bits per constant in the LP). Observe that the bit complexity of the LP you wrote is polynomial in |$\mathcal{GG}$| and c, \textbf{regardless of the subsets $S_p$ given as input}.
    \item  Denote by x an upper bound on the bit complexity of the LP you wrote, for any subsets $S_p$. Say that you have a $2^{-y}$-approximate Nash equilibrium, with y > x. What is an obvious choice of $S_p$ that would give your LP a value of at most $2^{-y}$? Observe that this same LP must in fact have value 0, and therefore solving it will yield an exact Nash equilibrium.
    
\end{enumerate}

\subsection{Proof}

The proof about problem 3 has referenced to chatGPT(The prompts file, \texttt{prompts.pdf}, will be submitted as appendix), and what I do is polishing its answer and write down here. The proof is as follows:

\paragraph{Step (a):}
To find an exact Nash equilibrium when given two subsets of strategies $S_A$ and $S_B$, we can set up a linear program (LP) as follows:

Let $x_i$ be the probability that player A plays strategy $i \in S_A$, and $y_j$ be the probability that player B plays strategy $j \in S_B$.

Objective function:
Minimize $0$, subject to the constraints:

\begin{enumerate}
    \item For each $i \in S_A$, $\sum_{i \in S_A} x_i = 1$, and $0 \le x_i \le 1$.
    \item For each $j \in S_B$, $\sum_{j \in S_B} y_j = 1$, and $0 \le y_j \le 1$.
    \item For each $i \in S_A$ and $j \in S_B$, $x_i \geq \sum_{j \in S_B} P_{ij} y_j$ (where $P_{ij}$ is the payoff of player A when playing strategy $i$ against player B's strategy $j$).
    \item For each $j \in S_B$ and $i \in S_A$, $y_j \geq \sum_{i \in S_A} P_{ij} x_i$. (here $P_{ij}$ is the payoff of player B when A playing strategy $i$ against B's strategy $j$)
\end{enumerate}

Solving this LP will give us the exact Nash equilibrium probabilities for player A and player B.

\paragraph{Step (b):}
To extend the result to polymatrix games, we can set up an LP for each player $p$ with strategies in the subset $S_p$. The LP will have variables $x_{pi}$ for each $i \in S_p$, representing the probabilities of player $p$ playing strategy $i$.

The objective function remains the same (minimize $0$), and the constraints are as follows:


\begin{enumerate}
    \item For each $i \in S_p$, $\sum_{i \in S_p} x_{pi} = 1$, and $0 \le x_{pi} \le 1$.
    \item For each $i \in S_p$ and $j \in S_{-p}$ (strategies of other players), $x_{pi} \geq \sum_{j \in S_{-p}} P_{pij} y_{pj}$ (where $P_{pij}$ is the payoff of player $p$ when playing strategy $i$ against other players' strategies $j$).
\end{enumerate}

Solving these LPs for all players will give us the exact Nash equilibrium probabilities for the polymatrix game.

\paragraph{Step (c):}
Now, we want to find the smallest $\epsilon$ such that there exists an $\epsilon$-approximate Nash equilibrium. We can modify the objective function of the LP to minimize $\epsilon$, subject to the same constraints as in Step (b). The LP will look like this:

Objective function:
Minimize $\epsilon$, subject to the same constraints as in Step (b).

Solving this LP will give us the smallest $\epsilon$, which represents the approximation error of the $\epsilon$-approximate Nash equilibrium.

\paragraph{Step (d):}
The bit complexity of the LP we constructed in Step (b) or (c) is polynomial in $|\mathcal{GG}|$ and $c$, regardless of the subsets $S_p$ given as input. In my opinion, This is because the number of constraints and the number of bits per constant in the LP are fixed for a given polymatrix game. (I just don't have a good idea to prove this part.)

\paragraph{Step (e):}
Suppose we have a $2^{-y}$-approximate Nash equilibrium, where $y > x$ (as defined in Step (d)). Since $y > x$, the approximation error $2^{-y}$ is smaller than the bit complexity upper bound $2^{-x}$.

To ensure that the LP we constructed in Step (c) has a value of at most $2^{-y}$, we can set the objective function to minimize $\epsilon$ (as in Step (c)) and solve the LP.

However, since the approximation error $2^{-y}$ is smaller than the bit complexity upper bound $2^{-x}$, the LP cannot have a value greater than $2^{-y}$. Thus, the LP will have a value of $0$, and solving it will yield an exact Nash equilibrium.

Since we have found an exact Nash equilibrium, this contradicts the assumption that the original game had a $2^{-y}$-approximate Nash equilibrium with $y > x$. Therefore, the only possibility is that there exists an exact Nash equilibrium for any polymatrix game with payoffs represented exactly using $c$ bits.

Q.E.D.


\newpage

\bibliographystyle{plain}
\bibliography{bibliography.bib}
\end{document}