\documentclass{article}
\usepackage{enumerate}
\begin{document}
\begin{abstract}
    In this paper, we address the problem of fair division of indivisible goods among multiple agents with general valuations. Envy-freeness, a widely studied notion of fairness, is not always achievable in this context. Instead, we consider a weaker notion called "envy-freeness up to any good" (EFX), where no agent envies another after the removal of any single good from the other agent's bundle.
    
    Our main result establishes the existence of an allocation satisfying EFX along with two additional desirable properties. Specifically, we demonstrate the existence of a partition of goods into $n + 1$ subsets $(X_1, \ldots, X_n, P)$, where each $X_i$ represents the bundle allocated to agent $i$, and the set $P$ remains unallocated (or donated to charity). The allocation possesses the following characteristics:
    \begin{itemize}
        \item Envy-freeness up to any good,
        \item No agent values $P$ higher than her own bundle, and
        \item The number of goods going to charity is strictly less than $n$, i.e., $|P| < n$ (with $m \geq n$ typically).
    \end{itemize}
    
    Our constructive proof leads to a pseudo-polynomial time algorithm for finding such an allocation. Notably, when agents have additive valuations and $|P|$ is large (close to $n$), our allocation guarantees a good maximin share (MMS). Moreover, a minor variation of our algorithm yields an allocation that achieves a $\frac{4}{7}$ groupwise maximin share (GMMS), a stronger notion of fairness compared to MMS. This result significantly improves upon the existing best bound of $\frac{1}{2}$ for an approximate GMMS allocation.
    
    This version of the paper extends beyond the preliminary version published in SODA 2020 [1] in two key aspects:
    \begin{enumerate}
        \item The pseudo-polynomial algorithm remains effective even when agents have general valuation functions (not restricted to gross-substitute valuations).
        \item We introduce a relaxed definition of the "most envious agent," contributing to a better understanding of the problem.
    \end{enumerate}
    \end{abstract}
    
\end{document}
