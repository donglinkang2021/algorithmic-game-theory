\documentclass{article}

\begin{document}

\section{Introduction}

Fair division of goods among competing agents is a fundamental problem in Economics and Computer Science. There is a set $M$ of $m$ goods, and the goal is to allocate goods among $n$ agents in a fair way. An allocation is a partition of $M$ into disjoint subsets $X_1, \ldots, X_n$ where $X_i$ is the set of goods given to agent $i$. When can an allocation be considered “fair”? One of the most well-studied notions of fairness is Envy-freeness. Every agent has a value associated with each subset of $M$, and agent $i$ envies agent $j$ if $i$ values $X_j$ more than $X_i$. An allocation is envy-free if no agent envies another. An envy-free allocation can be regarded as a fair and desirable partition of $M$ among the $n$ agents since no agent envies another; as mentioned in \cite{reference26}, such a mechanism of partitioning land dates back to the Bible.

Unlike land, which is divisible, goods in our setting are indivisible, and an envy-free allocation of the given set of goods need not exist. Consider the following simple example with two agents and a single good that both agents desire: one of the agents has to receive this good, and the other agent envies her. Since envy-free allocations need not exist, several relaxations have been considered.

\bibliographystyle{plain}
\bibliography{references}

\end{document}
