\documentclass{article}
\usepackage{amsmath, amssymb, amsthm, amsfonts}

\begin{document}

\section{Other Results}

In this section, we discuss additional findings concerning our EFX allocation with an approximate Minimum Marginal Surplus (MMS) guarantee. We observe that when the number of unallocated goods in our allocation is large, a corresponding increase in the number of sources (unenvied agents) is necessary. Furthermore, in our proposed EFX allocation, no agent envies the set of unallocated goods.

Let us consider the case where $|P| = n - 1$, implying that every agent is a source. As a result, no agent envies the bundle of any other agent, including the set of unallocated goods. For each agent $i$, the following inequality holds:

\[
v_i(X_i) \geq \frac{v_i(M)}{n + 1} \geq \left(1 + \frac{1}{n}\right)^{-1} \cdot \frac{v_i(M)}{n} \geq \left(1 - \frac{1}{n}\right) \cdot \text{MMS}_i(n, M),
\]

where the constraint $\frac{v_i(M)}{n} \geq \text{MMS}_i(n, M)$ is valid for additive valuations. We present our result for approximate-MMS allocation and an improved bound for approximate-Generalized Minimum Marginal Surplus (GMMS) allocation in the section titled "Guarantees on Other Notions of Fairness."

\end{document}
