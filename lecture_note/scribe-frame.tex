\documentclass{article}
\usepackage{amsmath, amssymb, amsthm, amsfonts}
\usepackage{fullpage}
\usepackage{graphicx}

\newtheorem{theorem}{Theorem}
\newtheorem{lemma}{Lemma}
\newtheorem{corollary}{Corollary}
\newtheorem{definition}{Definition}
\newtheorem{proposition}{Proposition}
\newtheorem{procedure}{Procedure}
\newtheorem{construction}{Construction}
\newtheorem{example}{Example}
\newtheorem{remark}{Remark}
\newtheorem{claim}{Claim}

\newcommand{\Rea}{{\mathbb R}}
\newcommand{\Int}{{\mathbb Z}}
\newcommand{\Rat}{{\mathbb Q}}
\newcommand{\Cmp}{{\mathbb C}}
\newcommand{\Nat}{{\mathbb N}}

\setlength{\oddsidemargin}{.25in}
\setlength{\evensidemargin}{.25in}
\setlength{\textwidth}{6.25in}
\setlength{\topmargin}{-0.0in}
\setlength{\textheight}{8.9in}

\renewenvironment{proof}{\noindent{\bf Proof:} \hspace*{1mm}}{
	\hspace*{\fill} $\Box$ }
\newenvironment{proof_of}[1]{\noindent {\bf Proof of #1:}
	\hspace*{1mm}}{\hspace*{\fill} $\Box$ }
\newenvironment{proof_claim}{\begin{quotation} \noindent}{
	\hspace*{\fill} $\diamond$ \end{quotation}}

\newcommand{\handout}[6]{
   \renewcommand{\thepage}{#1-\arabic{page}}
   \noindent
   \begin{center}
   \framebox{
      \vbox{
    \hbox to 5.78in { {\bf #2} \hfill #3 }
       \vspace{4mm}
       \hbox to 5.78in { {\Large \hfill #4  \hfill} }
       \vspace{2mm}
       \hbox to 5.78in { {\it #5 \hfill #6} }
      }
   }
   \end{center}
   \vspace*{4mm}
}
\newcommand{\lecture}[3]{\handout{#1}{6.896 Topics in Algorithmic Game Theory}{#2}{Lecture #1}{Lecturer: Constantinos Daskalakis}{Scribe: #3}}



\begin{document}

%header --- replace with appropriate values
\lecture{1}{February 3, 2010}{John Doe}

%start notes here
\section{Preliminaries}

A two-player game is formally defined as follows.

\begin{definition}
A {\em 2-player game} is defined by a pair of $m \times n$ payoff matrices $(R, C)$, whose rows correspond to the strategies of one of the players of the game, called the {\em row player}, and whose columns correspond to the strategies of the other player, called the {\em column player}. The strategy sets of the row and column players are identified respectively with the sets $[m]:=\{1,\ldots,m\}$ and $[n]:=\{1,\ldots,n\}$.
\end{definition}

\noindent The following theorem was established by John Nash in 1951~\cite{Na}.

\begin{theorem}
Every game has a Nash equilibrium.
\end{theorem}



%%%%  Bibliography goes here

\begin{thebibliography}{alpha}

\bibitem{Na} J.~Nash.
\newblock Noncooperative Games.
\newblock {\em Annals of Mathematics}, 54:289--295, 1951.

\end{thebibliography}


\end{document}